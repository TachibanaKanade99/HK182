\documentclass[10pt]{article}
\usepackage{pictex,amsmath,amsfonts,amssymb,amsthm,verbatim}
\usepackage{fullpage}
\usepackage{fullpage}
\usepackage{fancyhdr}
\usepackage{algorithm,algorithmic}
\usepackage{multirow}
\usepackage{gensymb}
\usepackage{mathrsfs}

\setlength{\voffset}{-0.25in}
\setlength{\headsep}{+0.5in}
\setlength{\parskip}{1em}
\setlength{\parindent}{0em}

\def\vu{\mathbf{u}}
\def\vv{\mathbf{v}}
\def\vb{\mathbf{b}}
\def\vw{\mathbf{w}}
\def\vs{\mathbf{s}}

\usepackage{xcolor}
\usepackage{titlesec}
\usepackage{mdframed}
\usepackage[utf8]{vietnam}\newmdenv[linecolor=blue,skipabove=\topsep,skipbelow=\topsep,leftmargin=5pt,rightmargin=-5pt,innerleftmargin=5pt,innerrightmargin=5pt]{mybox}

\newcommand{\quotes}[1]{``#1''}
\usepackage{minted}
\usepackage{graphicx,graphics}

\begin{document}

\begin{center}
	DATABASE SYSTEM
\end{center}

\section{Chapter I: Introduction to Database}

\subsection{Some Definitions: }
\begin{itemize}
	\item Database is a collection of related data.
	\item Data is facts that can be recorded and have implicit meaning.
	\item Database Management System (DBMS) is a computerized system that enables users to create and maintain a database.
	\item DBMS is a \textbf{general-purpose software system} that faciliates the processes of \textit{defining, constructing, manipulating} and \textit{sharing} databases among various users and applications.
\end{itemize}
	
\subsection{DBMS System:}
	\begin{itemize}
		\item \textbf{Defining the database} involves specify the data types, structures, and constraints of the data to be stored in the database.
		\item \textbf{Meta-data}, The database definition or descriptive information is also stored by the DBMS in the form of a database catalog and dictionary.
		\item \textbf{Constructing the database} is the process of storing the data on some storage medium that is controlled by the DBMS.
		\item \textbf{Manipulating the database} includes functions such as querying the database to retrieve specific data, updating the database to reflect changes in the miniworld.
		\item \textbf{Sharing the database} allows multiple users and programs to access the database simultaneously.
	\end{itemize}

\subsection{Application Program:}
	\begin{itemize}
		\item An \textbf{application program} accesses the database by sending queries or requests for data to DBMS.
		\item A \textbf{query} typically causes data to be retrieved.
		\item A \textbf{transaction} causes data to be read and written into the database.
	\end{itemize}

	- DBMS also provides functions for \textbf{protecting} and \textbf{maintaining} the database system:
	\begin{itemize}
		\item \textbf{Protecting} includes \textit{system protection} against hardware or software malfuncton (or crash) and \textit{security protection} against unauthorized and malicious access. 
		\item A DBMS need to \textbf{maintain} the database system by allowing the system as requirement change overtime.
	\end{itemize}

\bigbreak
\begin{mybox}
	\begin{center}
		Database system = database + DBMS
	\end{center}
\end{mybox}
	
	\bigbreak
	\begin{itemize}
		\item Conceptual Design.
		\item Logic Design.
		\item Physical Design.
	\end{itemize}

	- Design of the new application for an existing database or design of a brand new database starts off with a phase called \textbf{requirements specification and analysis}. \\

	- These requirements are documented in detail and transformed into a \textbf{conceptual design} that can be represented and manipulated by some computerized tools $\rightarrow$ easily modified, maintained and transformed into \textbf{database implementation}. \\

	- The design is then translated into \textbf{logical design}. that can be expressed in a data model implemented in a commercial DBMS. \\

	- The final stage is called \textbf{physical design}, further specifications are provided for storing and access database. \\

\subsection{Characteristic of the Database Approach: }
\begin{enumerate}
	\item File processing:
	\begin{itemize}
		\item A traditional \textbf{file processing}, each user defines and implements the files needed for the specific software application.
		\item Both users are interested in same data but each users maintain separate files and programs to manipulate files.
		\item This redundancy in defining and storing data results in wasted storage sapce and in redundant effort to maintain common up-to-date data.
	\end{itemize}

	\item Database Approach:
	\begin{itemize}
		\item In database approach, a single repository maintains data that defined once and then accessed by various users repeatedly though queries, transactions, and application programs.
		\item The main characteristic of the database approach vs file processing:
		\begin{itemize}
			\item Self-describing nature of a database system.
			\item Insulation (ngăn cách) between programs and data, and data abstraction.
			\item Support of the multiple views of the data.
			\item Sharing of data and multiuser transaction processing.
		\end{itemize}
	\end{itemize}

	\begin{enumerate}
		\item Self-Describing Nature of a Database System:
		\begin{itemize}
			\item The database system contain not only the database itself but also a complete definition or description of database structure and constraints.
			\item The information stored in the DBMS catalog is called \textbf{meta-data}, it describles the structure of the primary datatabase.
			\item NOSQL systems do not require meta-data, those data is sstored in \textbf{self-describing data} that includes the data item names and dara values together in 1 structure.
			\item The DBMS software must work equally well with \textit{any number of database applications}
		\end{itemize}

		\item Insulation between Programs and Data, and Data Abstraction:
		\begin{itemize}
			\item The structure of data files is stored in the DBMS catalog separately from the access program, this is called \textbf{program-data indepedence}. \\
			(Why \textbf{meta-data enable data-program independence ?})
			\item An operation (known as function or method) is specific in \textit{interface} and \textit{implementation}.

			\begin{itemize}
				\item interface includes operation name and its data types of its argument.
				\item implementation of the operation is specified separately and can be changed without affecting the interface.
			\end{itemize}

			\item This may be known as \textbf{program-operation independence}.
			\item The characteristic that allows program-data independence and program-operation independence is called \textbf{data abstraction}.
			\item A \textbf{data model} is a type of data abstraction that is used to provided this conceptual representation. The data model \textit{hides} storage and implementation out of database users.  
		\end{itemize}

		\item Support of multiple views of the Data: 
		\begin{itemize}
			\item A multiuser DBMS whose users have a variety of distinct applications must provide facilities for multiple views.
		\end{itemize}

		\item Sharing of Data and Multiuser Transaction Processing: 
		\begin{itemize}
			\item This is essential if data for multiple users to be integrated and maintained for a single database.
			\item The DBMS must include \textbf{concurency control} software to ensure that several users trying to update the same data so that the results of update is correct.
		\end{itemize} 
	\end{enumerate}

	\item Actors on the Scene:
	\begin{itemize}
		\item Database Adminstrators.
		\item Database Designers.
		\item End users.
	\end{itemize}
\end{enumerate}

\subsection{Advantages of Using the DBMS Approach}
\begin{enumerate}
	\item Reduce Redundancy:
	\begin{itemize}
		\item Data normalization: All logical data item are stored in \textit{only one place} in the database.
	\end{itemize} 

	\item Restricting Unauthorized Access:
	\item Provide persistent storage for Program Object:
	\item Provide Storage Structures and Search Techniques for efficient Query processing:
	\item Provide Backup and Recovery:
	\item Provide Multiple User Interfaces:
	\item Represent Complex Relationships among Data:
	\item Enforcing Integrity Constraints:
	\item Permitting Interfencing and Actions using Rules and Triggers:
\end{enumerate}

\subsection{When not to use DBMS}
\begin{enumerate}
	\item It maybe more desirable to develop customized database applications:
	\begin{itemize}
		\item Simple, well-defined database applications that are not expected to change at all.
		\item Stringent, real-time requirements for some application programs that may not met bacase of DBMS overhead.
		\item Embedded devices with limited storage capacity, where a general-purpose DBMS would not fit.
		\item No multiple user-access data.
	\end{itemize}
\end{enumerate}

\bigbreak
\section{Chapter II: Database System Concepts and Architecture}

\subsection{Data Models, Schemas, and Instances}
\begin{enumerate}
	\item Data Models:
	\begin{itemize}
		\item A data model is a collection of concepts that can be used to describle the structure of the database - provides the neccessary means to achieve this abstraction.
		\item Most data models also include set of \textbf{basic operations} for specifying retrievals and updates on the database.
		\item Types of Data Model:
		\begin{itemize}
			\item \textbf{High-level} or \textbf{conceptual data models} provide concepts that are close to the way many users realize data.
			\item \textbf{Low-level} or \textbf{physical data model} provide concepts that describle the detail of how data is stored in the conputer storage media, typically magnetic disk.
			\item \textbf{Representational} (or \textbf{implementation}) \textbf{data models} provide concepts that may be easily understood by end users but that are not \underline{too far removed} (very different from) from the way data is organized in computer storage.
		\end{itemize}
	\end{itemize}

	\item Schemas, Instances, and Database State:
	\begin{itemize}
		\item Schemas is kind of layout of the database.
		\item The actual data in the database may change quite frequently. The data in the database at the particular moment of time is called a \textbf{database state} or \textbf{snapshot}
	\end{itemize}
\end{enumerate}

\subsection{Three-Schema Architecture and Data Independence}
\begin{enumerate}
	\item The three-schema Architecture: \\
	
	\includegraphics[scale = 0.5]{hinh.png}

	\begin{itemize}
		\item The \textbf{internal level} has an \textbf{internal schema} describles the physical storage structure of the database.
		\item The \textbf{conceptual level} has a \textbf{conceptual schema} describles the structure of the whole database for users.
		\item The \textbf{external} or \textbf{view level} includes a number of \textbf{external schemas} or \textbf{user views}.
	\end{itemize}

	\item Data Independence: 
	\begin{enumerate}
		\item \textbf{Logical data independence:} is the capacity to change the conceptual schema without having to change external schemas or application programs.
		\item \textbf{Physical data independence:} is the capacity to change the internal schema without having to change the conceptual schema
		\item Summary: Data indepedence occurs when the schema is changed at some level, the schema at the next higher level remains unchanged; only the \textit{mapping} between the teo levels is changed. Hence, aplication referring to the higher level schema do not need to be changed.

		\bigbreak
		\includegraphics[scale = 0.5]{hinh1.png}
		\bigbreak 
	\end{enumerate}
\end{enumerate}

\subsection{Database Languges and Interfaces}
\begin{enumerate}
	\item DBMS Languages:
	\begin{itemize}
		\item \textbf{Data definition language} (DDL) : used to define both conceptual and external schemas where no strict separation of levels is maintained.
		\item \textbf{Storage definition language} (SDL):
		\begin{itemize}
			\item is used to specify the internal schema.
			\item DDL is used in conceptual schema.
			\item It is used where there is a clear separation between the conceptual and external levels
		\end{itemize}  
		\item \textbf{View definition language} (VDL) : is used to specify user views and their mapping to the conceptual and external schemas.
		\item Beside, DBMS also provides a set of operations or a language called the \textbf{data manipulation language} (DML) for manipulating the database. \\

		\begin{itemize}
			\item A \textbf{high-level} or \textbf{nonprocedural} DML.
			\begin{itemize}
				\item It is used to specify complex database operations concisely (chính xác).
				\item Many DBMSs allow high-level DML statements to be entered from the display monitor or terminal or to be embedd in a general-purpose programming language.
				\item High-level DML, such as SQL can specify and retrieve many records in a single DML statement, it is called \textbf{set-at-a-time} 
			\end{itemize}
			\item A \textbf{low-level} or \textbf{procedural} DML.
			\begin{itemize}
				\item It \textit{must} be embedded in a general-purposes language.
				\item It is also called as \textbf{record-at-a-time} since it functions related to retrieving and processing individuals records and objects.
			\end{itemize}
		\end{itemize}

		\item Whenever DMLS commands, whether high-level or low-level, are embedded in a general-purpose language and that language is called the \textbf{host language} and DML is called the \textbf{data sublanguage}.
		\item A high-level DML used in a standalone interative manner is called a \textbf{query lenguage}.
	\end{itemize}

	\item DBMS Interfaces:
	\begin{itemize}
		\item Menu-based Interfaces for Web Clients or Browsing
		\item Apps for Mobile Devices
		\item Forms-based Interfaces
		\item Graphic User Interfaces
		\item Natural Language Interfaces
		\item Keyboard-based Database Search
		\item Search Input and Output
	\end{itemize}
\end{enumerate}

\subsection{The Database System Environment}

 




\end{document}